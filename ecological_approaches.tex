\documentclass[12pt,draft,letter]{article}
\usepackage{url}
\usepackage{natbib}
\usepackage{endnotes}
\title{Interactional Theories of Cities}
\author{Forest Gregg}

\let\footnote=\endnote
\let\cite=\citep


\usepackage{todonotes}

\begin{document}
\maketitle
\noindent This document has \makeatletter\@@input|"echo `texcount -1 -inc ecological_approaches.tex`| cut -c1-4"\makeatother words
\section{Introduction}
Urban sociologists have used, mainly, three ecological principles to
explain why things are where they are: resource dependence, competition,
and mutualism.

The first principles is that peoples and organizations need some
resources which are spatially fixed and rare. In order to access those
resources, people and organizations will locate near them. For
example, mining towns appear near mineral deposits and high tech
businesses form in cities with a highly educated workforce.

Where there are rare resources, there is likely to be competition and
this is the second common principle. The presence of particular
peoples or organizations is not just due to the presence of resources,
but the succesful and exlusionary competition for those resources.
For example, almost all retail businesses would like to locate in the
place of maximum accessibility, but only certain businesses can afford
the rent.

Finally, the mutualism emphasizes how people and
organizations form the environments for one another. Jeweler's locate
near each other because the benefits for of forming a district that
attracts customers who want to search and compare goods outweighs the
losses competition. Industries cluster geographically because
productions requires the coordinated inputs of different
firms. Entertainment districts, and other scenes, emerge because
certain businesses attract the types of people who want to enact a
certain lifestyle and those people form the consumer base for certain types
of businesses to prosper.

Within the discipline of ecology, interactions between species are
often characterized by the net effect of the presence of one species
upon another. For one species, say wolves, the presence of plentiful
rabbits increases the number of wolves, plentiful hunters decreases
wolves, and plentiful morning glories has no effect. A pair of species
can have one of six types of interactions show in Table~\ref{tab:ecological}

\begin{table}[h]
\centering
\begin{tabular}{lcccl}
  & +         & 0                        & -           &  \\
+ & Mutualism & Resource Dependence      & Antagonism  &  \\
0 &           & Neutralism               & Amensalism  &  \\
- &           &                          & Competition & 
\end{tabular}
\caption{Ecological Interactions}
\label{tab:ecological}
\end{table}

In this paper, we will explore how sociologists have used these types
of interactions to understand the ``spatial and temporal distribution
of social and cultural data'', \footnote{Gettys} particular within and
among cites.  We will focus on resource dependence, competition, and
mutualism.  The remaining three types of interactions have been little
used by urban sociologists, but I will touch on a few examples.

Our discipline has mainly explained cities and urban form through
resource dependence and competition for scarce resources. However,
since the 1990s, mutualism has emerged as the major 
explanation for the rennaissaince of American central cities
cities. However, the full logic of this turn has not been developed,
and it is here where I will spend the bulk of the paper.


\section*{Competition} 
In the late 19th century and early 20th century, American urban
sociologists grappled with the unprecedented size and rapid growth of
the industrial city. Their main explanation was that economic
geography of transportation costs selected firms located at certain site;
sites that would swell into great cities.

Technological developments in shipping, canal building, and the rail
line had made long-distance transport much cheaper, but short-distance
transport remained expensive. Firms located near the central nodes of
the long-distance transport networks had much lower total costs of
transport than their backwater competitors. Economic competition
between firms in the newly national market would, therefore, tend to
select firms located at the best sites. As firms concentrated into a
few sites, their employees--the city's population--would follow.

In equilibrium, economic competition should minimize the total costs
of transportation for the entire economy.  Within a city, economic
competition would also result in the most accessible, central places
being occupied by the organizations that most benefited from
access the city.

In this model, there is no necessary relation between the competitive
interactions between firms and urban agglomeration. Competition
should only lead to concentrations when the economic geography
strongly favors concentration. When the particular complex of
transportation technology, population distribution, manufacturing
technique, and economic integration that fostered the dense industrial
city passed, so did competition as an explanation for urban
concentrations.

\subsection*{Distributions of Settlements and Economic Geography}
Sociologists have mainly left geographers to the task of explaining
the size and locations of settlements and geographers have
historically focused on two factors: labor costs and transportation
costs of raw material and finished goods.\footnote{Firey 24, what's
  his face in the City}

The geographic distribution of raw materials, transportation network,
population, and labor prices combine to form a geography of costs of
operating a firm. Within this economic geography, certain locations
confer advantage to their resident firms who win out against their
poorly positioned competitors. Advantaged locations would swell while
the backwaters dwindled.\footnote{ullman}

Not all firms would prefer the exact same type of location. Firms that
transformed bulky raw material into much lighter finished goods should
locate near the source of raw materials. Firms that distributed
perishable goods should locate near their consumers. Whatever the type
of firm though, the distribution of people and raw materials and the
transit network produced locations that would place that type of firm
at a competitive advantage.

The early industrial city grew so quickly because the network of first
canals and then railroads cut, deeply, the costs of transportation the
nodes of that network. Manufacturing firms located at those nodes
could outcompete firms located in bypassed towns and even local craft
works. Because short-distance haulage remained expensive, firms
clustered around the transit nodes, and workers clustered around the
firms.\footnote{hawley}

\subsection*{Distributions within Settlements}
Within cities, short-distance transportation costs also create an economic
geography. Across a city, the distribution of people and organizations
interacts with the transportation networks to produce areas of greater
accessibility. The most accessible place should be dominated by the
type of users who benefit most from accessibility relative to their
unit costs of land.\footnote{Alonso} The unit costs, in turn, were the
outcome of market competition between firms and residents for the
accessibility.

In a rare point of consensus, the early 20th century urban
sociologists agreed that commercial firms should and could outbid
residents or other firms for the most accessible parts of the
city..\footnote{burgess, chauncy and ullman}. Even Firey allowed that
it was economically rational for large department stores to dominate
downtowns.

Beyond the central business district, there were many locally maximal
accessible locations, typically at the intersection of major streets
or transportation lines. Smaller, local commercial firms should occupy
these spaces, forming miniature central business district.\footnote{hoyt}

Manufacturers could also form distinct nuclei of populations. Oriented
to national markets, they would choose locations that minimized their
long distance transport costs and had ample land for large
plants. These locations were often far from the center of population,
but if these large employers were concentrated, they could attract
employees to reside in their proximity.

While sociologists more or less agreed that competition for
accessibility would determine the locations of commerce and industry,
there was substantial disagreement about how people should distribute
themselves in the places left to them.

\subsection**{Population Ecology}
The competitive theories of industrial city are special case of the
population ecology model, most associate with Hannan and Freeman.  In
economic geography, the \emph{spatial distribution} of firms is the
result of the competition and selection of firms who have a
\emph{locational} advantage over competitors facing the same
market. In the classic formulation of population ecology, the
\emph{distribution of populatoins} of different types of firms is the
result of competition and selection of firm who have a
\emph{strategic} advantage over competitors facing the same market. In
both cases, the distribution of firms is the result of selection
among competitors by an environment.

\subsection*{Spatial Dynamics of Competitive Interactions}
As a field, we have mainly attended to competitive interactions as a
component mechanism for how an environment selects
populations. Competitive interactions, though, do have their own
spatial dynamic. All else being equal, competitive interactions should
lead to the spatial dispersion of populations, as individuals will
benefit from isolating themselves from competitors.

In a competitive interaction, group A would be better off if group B
was removed from its presence, and vice versa. Competing groups would
be best off if they were physically separated.  In real estate
markets, firms competing for accessibility bids up the land rent of
the central place leading most organizations and people to live
elsewhere, where the accessibility is lower but so is price.

If a particular site provide advantages to populations that locate
there, competition both explains why the population concentrates at
that site--they are better able to reproduce or survive than
population located elsewhere. Competition also explains why the entire
population doesn't concentrate as close as they can to a single
point. At some level of concentration, the disadvantages of competition
outweigh the initial advantages of the site.

In economic terms, this is the same idea, from Alonso, that
competitive farmers should all make the same ``normal profit''.
Regardlesss of the fertility or location of the land, since more
fertile land, closer to market will be priced higher than marginal
distant land.

Given an economic geography that provides a fixed economic advantage
for populations of firms at a city site, competition should explain
city size. The advantages of the site produces concentration but the
effects of competition limit concentration. If those initial
advantages of place disappear, the centrifugal dynamic of competition
should quickly reduce concentration.

The initial advantages of the industrial city eroded throughout the
20th Century. First, the automobile and then the highway system
reduced the costs of short-distance transport, allowing manufacturing
firms to move away from the depots of long-distance transport. The
early communication and transportation innovations of the industrial
age, the telegram and the railroad, reduced the costs of
transportation and produced a national or continental market for
manufacturing goods. Further developments in communication and
transportation, particularly containerization, reduced the costs of
international transport and put manufacturing firms in global
competition. American industrial cities no longer provided a relative
advantage to manufacturing firms, and the populations of firms
dwindled. The population of people did as well.







\section*{Where do the People Go}
In the classic models, the distribution of natural resources, market
populations, and transportation networks form an geography of better
or worse places to locate business functions. The best locations will
be occupied by businesses, which will, in turn produce a geography of
commute times for residents.

In the early industrial period, when walking was the main mode of
transportation, this geography hemmed growing population to within a
few miles of their places of employment. Within a century, though,
rapid improvements in short distance transportation had relaxed the
geographic connection between work and homes.\footnote{Hawley 90}. By
the 1920s, the major American theorists of urban form recognized that
the least accessible locations, the edges of the city, contained the
most attractive residential locations.\footnote(burgess, 53, hoyt}

Unlike firms, households did not seem to be choosing locations to
minimize the cost of land and transportation. A different type of
theory was necessary to explain residential choice, and for the most
part, early 20th century theorists put forward theories of amensalism
between different types of uses and people. Ecologically, amensalism 
is when the presence of group A harms group B but the presence of
group B does not effect group A.

In Ernest Burgess's famous concentric zone theory, in an expanding
city, the central business district will expand into a zone of working
class residences. Through population growth or displacement by the
expanding central business district, working class residents encroach 
into a zone of wealthier residences, who move further away to avoid
their new neighbors.\footnote{burgess 50}

Homer Hoyt follows the same line when he observed that a growing
population of immigrants and non-whites in Chicago promoted the
expansion of the city because these groups ``forced or stimulated the
old American stock to see new neighborhoods and has caused them to
migrate from their old homes.''\footnote{hoyt, 317}

Does hoyt have another explanation for why rich and poor are
segregated besides antipaty of the rich for the poor?

Hoyt, of course, had additional theory of why 

\todo{The concept of filtering actually seems more antagonistic}
\todo{how do we think of shelling}







\local transit line as factor in sectoral theory

H

\section*{Mutualism}
If the survival of firms depends on the place based costs of
production and transportation, then changes in the geography of those
costs will change the distribution of those types of firms. In the
early industrial period, transportation costs dominated other
locational factors, leading to the concentration of industry in key
nodes of the transportation. After the second world war, rising wages,
trucking, containerization, and new manufacturing techniques
undermined earlier advantages of the industrial city. Total
transportation costs became a secondary or tertiary consideration, and
the manufacturing firms first left the city often, later, the country.

For the theorists of the industrial city, the locational efficiencies
for firms were the fundamental reason for the city and its
growth. When the city was no longer efficient for longer efficient,
the city should dissipate. Indeed, it seemed to be doing just that. 

However, by the 1980s, the growth of the corporate service sector and
the return to the city of high status groups confounded expectations
and demanded new explanations of urban form. Two main theories have
developed. First, that some industries require in-person coordination
between firms in order to thrive, so firms location near one
another. Second, some people live in cities in order to enact
lifestyles that are particularly available in cities. 

As we will see, these are both theories of mutualism. Mutualism is
where the presence of group A benefits group B and the presence of
group B benefits group A. 

\subsection*{Clustering and Craft}
Within urban sociology, the most prominent proponents of firm-level
mutualism have been Saskia Sassen and Michael Porter.

Saskia Sassen's work on global cities makes two basic moves. First,
she extends the Donald Bogue's concept of metropolitan dominance to a
globalized economy. As the size and scale of international production
expands, she argues, there is a complementary increase in centralized
administrative functions. Second, and going beyond Bogue, she argues
that those administrative functions should be concentrated in certain
sites because of these administrative functions are the joint product
of multiple firms, which requires intensive coordination facilitated
by proximity: ``At the current stage of technical development, having
immediate and simultaneous access to the pertinent experts is still
the most effective way to operate, especially when dealing with a
highly complex product.''\footnote{95}

Sassens argument about how proximity facilitates the production of
complicated and innovative services is a specific case of generic
argument that Michael Porter makes about ``industry clusters.''
Geographic clustering of an industry can, he argues, provide a number
of benefits for international competition. A concentration of
competitors, suppliers and customers can lead to specialization and
efficiency. Suppliers and competitors can are better able to
communicate and coordinate the production of new products and
services. Proximity lets industry participants better survey each
other and understand the overall state of the industry, allowing for
firms to more quickly react to changes and identify
opportunities.\footnote{157}

The economic engines of post-industrial cities are largely
administered like pre-industrial craft work. The work is variable,
creative, unique, and often customized for the client. As Stinchcombe
predicted for this type of work, the production is largely administed
along craft lines. \footnote{Stinchome} That is, the work is
accomplished by a network of specialist firms, that coordinate
extensively but preserve their own jurisdictions of work. That some of
these firms are enormous investment banks, consulting companies,
medical device companies, or automotive suppliers does not materially
change the craft-like nature of the administration of the work.

Given current communication technology, cities are good for craft-like
work. If the work can be reorganized for industrial production, the
city loses its advantage.

\subsection{Lifestyle}
Like earlier theorists, Sassen and Porter see firms as the prime
movers for the cities. However, other, new theories of cities have
elevated the resident as the driver of urban concentration and
growth. Early work in this line saw a new types of status groups
choosing city life as a form of consumption. Terry Clark and Dan
Silver's recent work goes beyond this to contemplate the co-production
of types of people and types of places. This is a mutualism between
people, between firms, and between people and firms.

Richard Florida's work has been typical of the main lifestyle
arguments. In his model, the workers the human capital necessary for
post-industrial production tend to like living places that are
diverse, tolerant, and creative. These workers choose a place to live
based upon those preferences, not primarily, upon where they can get a
job. High value firms and innovation and economic growth follow the
people.\footnote{richard florida, entertainmenet machine, sassen,
  berry}

This is not just a reversal of the firm-centric models of the
industrial city, it's is also incompatible of with the earlier
status-based theories of residential selection. Those status based
theories assumed that there was a relatively stable status order based
on race, ethnicity, and occupation (and maybe caste, tribe, or
religion in other countries). People might not express their
residential preferences in terms of status, but their choices
effectively revealed a status preference. The lifestyle arguments
contradict this, arguing that an important group of people want to
live where it's fun, exciting, diverse. If the city or neighborhood is
gritty--poor and rundown, then that can be good too.

The lifestyle argument does not reject the theory that people choose
where to live in order to minimize the social distance between
themselves and the population they wish to emulate; it argues that
status groups reduced to a linear order.

However, a theory that people choose where to live based upon
lifestyle preferences is not a theory of why people cluster in
cities. We need an additional link between geographic concentration
and lifestyle. For lifestyle to explain why people move to cities,
there must be an explanation of why cities are especially amenable to
the crucial lifestyle. 


Claude Fischer's subcultural theory provides one such account.
In this theory, cities stimulate expressive unconventionality because
of they stimulate subcultures. Fisher enumerates many mechanisms for
why cities generate and strengthen subculture and why a diversity of
subcultures, but they are first or second order effects size or
growth. In his account, unconventionality is an effect of
agglomeration not a cause.\todo{expand on this}

In their recent work on ``scenes'', Terry Clark and Daniel Silver
married a theories of subcultures and lifestyle based residential
choice to form the first, explicit theory of urban agglomeration
driven by lifestyle. In their account, all people in advanced
economies are increasingly choosing a place to live in order to choose
a scene where they can express their aesthetics. A place's scene is
made by the current residents, both directly and indirectly through
the retail institutions that serve residents' lifestyles. When more of the
right type of people are in a scene, there are more opportunities for
expressive coaction and more institutions serving as venues for
expressive action.

\subsection*{Mutualism and Space}
Mutualistic interactions 

\subsection*{Coherence}





\section*{Future Directions}
If mutualism is the prime driver of American urbanism, we have only
really begun to understand it's dynamics. Here are a number of
critical questions of whether and how mutualism will affect American
cities: how does mutualistic coherence affect development, where do
aesthetics come from, and what happens to people outside of the
mutualistic circle.

\section*{Detecting Coherence}
In the mutualistic theories of firms or people, there is an implicit
concept of a critical, coherent population. Above some threshold
number of complementary people or organizations, people see an area as
having a scene and can coordinate on that shared perception. Above
some threshold of complementary firms, there is industry cluster and
firms and can benefit from co-location. Below some threshold, there is
no there.

When a critical threshold is passed, the mutualistic feedback loop
should cause the proliferation of complementary populations. The death
rate should fall or the birth rate should increase or both. After
crossing a threshold, the concentration of complementary populations
should increase in proportion to the current population,
i.e. exponentially. Eventually, the forces of competition will limit
that growth either through price of land or the limits of client
demand.

Even if a scene or cluster is no longer growing, we should still
expect that the cluster will have different birth or survival rates
for complementary and noncomplementary populations compared to a
place with a different scene or cluster. Birth and survival also
include in and out migration from the place.

Critical coherence is both a interesting concept in it's own right and
a critical test for mutualistic theories. If we cannot detect expected
patterns of take-off after a certain threshold is reached, that will
be a serious blow to the ideas of industry clusters or scenes. If the
pattern does exist, then coherence thresholds will be a useful tool
for understanding the trajectory of cities and useful diagnostic for
local economic development.


\section*{Elective Affinities Between Post-Industrial Work and
  Aesthetics}

If people are attracted to places based upon aesthetic, lifestyle
considerations, then the origin of their aesthetics becomes a
critical question for urban sociology. In his discussion of how one's
environment acts on a person to form our aesthetic judgement, Bourdieu
is likely very helpful here. However, in his wider claims he argues
that at any particular time the space of taste should largely be
organized along two axes. This does not seem enough to describe the
many scenes that exist within and between cities. If Clark and
Silver's thesis is correct there are many more than two dimensions
along which the aesthetics of a place can vary.\footnote{Bourdieu is
  careful to not specify the aesthetic dimensions that taste vary
  upon, just that at any time the space of taste will be largely
  organized along two axes.\cite{bourdieu_distinction:_1986}}

Looking more closely at the connection between work and aesthetics
will likely be very useful for urban sociology. If there is an
elective affinity between different types of work and aesthetics, then
places with concentrations of particular types of work, industry
clusters, should express particular aesthetics, and vice versa. As
mutualistic feedback leads cities to specialize in different types of
industries, they should, as part of the same process, differentiate in
their aesthetics.

\subsection*{The Rest of Us}
In both the theories of industry clusters and scenes, the key actor
are very high skill workers who's labor is critical to post-industrial
production. These types of workers are minority of Americans and a
minority of city dwellers. How do the rest of us fair in a mutualistic
city, if we are not part of the virtuous cycle.

Saskia Sassen provides a partial answer. These highly paid workers
create substantial demand for low-skill service jobs. However, there
is a great more work to be done on how the political economy of places
where a coherent set of firms and workers generate most of the wealth,
but are not necessarily concentrated into large organizations. Should
we expect to see guild like behavior?

At the neighborhood level, many places don't have much of a
scene. Does living in a place without a distinct aesthetic coherence
have any predictable outcomes for the ability of a community to solve
problems or the welfare of its residents?



\newpage
\begingroup \parindent 0pt
\parskip 2ex
\def\enotesize{\normalsize}
\theendnotes
\endgroup

\newpage
\bibliographystyle{plainnat}
\bibliography{Ecology}


\end{document}
