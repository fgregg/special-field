\section*{Competition} 
In the late 19th century and early 20th century, American urban
sociologists grappled with the unprecedented size and rapid growth of
the industrial city. Their main explanation for these astonishing
social facts was that aggregations of people and organizations into
dense settlements and the locations of those settlements across the
land, were economically efficient. The distribution of people and
organizations among settlements tended to minimize the total costs of
transportation for the raw materials and finished goods of industrial
manufacturing. Within settlements, the central places should be
dominated by uses that most benefited from access the city or be
accessible by the city.

In these efficiency arguments, economic competition between firms was
the ultimate mechanism. Firms in places with cheap access to inputs
and markets survived and their less well placed competitors did
not. Populations followed jobs and settlements waxed or waned on the
fortunes of its firms competing in a national market.  In equilibrium,
the people are organizations are placed across and within settlements
in the manner that maximizes economic efficiency.

Contemporary critics attacked the competitive models on three
fronts. First, empirically, many districts of American cities were
used for purposes that were grossly economically inefficient. Second,
market competition for land is a historically contingent cultural
value. Some facts of American cities may be explained by market
efficiency, we need to attend to other values to adequately understand
urban forms. Third, we should not expect that distribution of people
and organizations in a growing city to be globally efficient. The form
of a growing city requires a theory of growth, not a theory of static
equilibrium. 

\subsection*{Distributions of Settlements and Economic Geography}
Sociologists have mainly left geographers to the task of explaining
the size and locations of settlements. The geographers' theories, or
at least the ones that we have attended to, have historically, focused
on two factors,: labor costs and transportation costs of raw material and
finished goods.\footnote{Firey 24}

The geographic distribution of raw materials, transportation network,
manufacturing technology, population, and labor prices combine to form
a geography of costs of operating a firm. Within this economic
geography, certain locations confer advantage to their resident firms
who win out against their poorly positioned
competitors. Advantaged locations would swell while the backwaters
dwindled.\footnote{ullman}

\subsubsection*{Population Ecology}
This economic geography model is a special case of the population
ecology model of Hannan and Freeman.  In economic geography, the
\emph{spatial distribution} of firms is the result of the competition
and selection of firms who have a \emph{locational} advantage over
competitors facing the same market. In the classic formulation of
population ecology, the \emph{distribution of populatoins} of
different types of firms is the result of competition and selection of
firm who have a \emph{strategic} advantage over competitors facing the
same market.\{cite here}. 

Both population ecology and the special case of economic geography are
models of how environments select firms that are somehow better fitted
to the environment. In these models, firms cannot meaningfully change
their environment. This is a very useful simplification and a valid
one for the small restaurants that Hannan and Freeman used as their
first test case.  In economic geography, this simplification may be
less valid, as the geographic concentration of firms is likely to
effect the distribution of people, the costs of labor, and even
transportation and manufacturing technique. We'll pick up this point
again when in the section on mutualistic interactions.

\subsection*{Distributions within Settlements}
Within cities, the transportation costs also create an economic
geography. Across a city, the distribution of people and organizations
interacts with the transportation networks to produce areas of maximum
accessibility. The most accessible place should be dominated by the
type of users who benefit most from accessibility relative to their
unit costs of land.\footnote{Alonso} The unit costs, in turn, were the
outcome of market competition between firms and residents for the
accessibility.

In a rare point of consensus, the early 20th century urban
sociologists agreed that commercial firms most benefited from
acccessibility, so should dominate the area of maximum
accessibility.\footnote{burgess, chauncy and ullman}. Even Firey
allowed that it was economically rational for large department stores
to dominate downtowns.

The central business district, the site of global maximum
accessibility, was the largest nucleus of firms it was not the only
one. There were many locally maximal accessible locations, typically
at the intersection of major streets or transportation lines. Smaller,
local commercial firms should occupy these spaces, forming miniature
central business district.

Manufacturers could also form distinct nuclei of populations. Oriented
to national markets, they would choose locations that minimized their
long distance transport costs and had ample land for large
plants. These locations were often far from the center of population,
but if these large employers were concentrated, they could attract
employees to reside in their proximity.

While sociologists more or less agreed that competition for
accessibility would determine the locations of commerce and industry,
there was substantial disagreement about how people should distribute
themselves in the places left to them.



\todo{note zoning and politics is completely absent from these
  accounts}



\todo{touch on macpherson's community matrix}
