\section*{Resource Dependence} 
In the late 19th century and early 20th century, American urban
sociologists grappled with the unprecedented size and rapid growth of
the industrial city. Their main explanation for these astonishing
social facts was that aggregations of people and organizations into
dense settlements and the locations of those settlements across the
land, were economically efficient. The distribution of people and
organizations both within and among settlements tended to minimize the
total costs of transportation of industrial goods.





 been both used in both explicit theories of
urban forms and an often implicit mode of analysis. Explicit theories
first identify a spatial distribution of resources that were fixed or
slow-changing, and describe how people and organizations will
distribute themselves to optimizes access to those resources. In the
implicit mode, the analyst makes no special attempt to distinguish a
fixed, or slow changing environment from the relatively quickly
changing people or organizations of interest. Instead the analyst
makes a ceteris parebus assumption that the distribution of resources
are fixed and shows how actors seem to be respond to assumed fixed resources.



\subsection*{Subdominance}

\subsection*{Population Ecology}

\subsection*{Creative Class}

Some of the first theories of The first theories of 

The earliest explanations of 

The earliest explanations for 


Among biological species, when species A benefits from the presence of
species B, but species B is not affected by actor A, they have a
"commensalistic" interaction. Examples are species which eat the
carcasses or refuse of another species. In other words, one species is
a resource for the the other without being much affected by it. 

Resource dependence is the most common form of explanation for the
spatial distribution of



