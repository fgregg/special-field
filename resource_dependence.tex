\section*{Competition} 
In the late 19th century and early 20th century, American urban
sociologists grappled with the unprecedented size and rapid growth of
the industrial city. Their main explanation for these astonishing
social facts was that aggregations of people and organizations into
dense settlements and the locations of those settlements across the
land, were economically efficient. The distribution of people and
organizations among settlements tended to minimize the total costs of
transportation for the raw materials and finished goods of industrial
manufacturing. Within settlements, the central places should be
dominated by uses that most benefited from access the city or be
accessible by the city.

In these efficiency arguments, economic competition between firms was
the ultimate mechanism. Firms in places with cheap access to inputs
and markets survived and their less well placed competitors did
not. Populations followed jobs and settlements waxed or waned on the
fortunes of its firms competing in a national market.  In equilibrium,
the people are organizations are placed across and within settlements
in the manner that maximizes economic efficiency.

Contemporary critics attacked the competitive models on three
fronts. First, empirically, many districts of American cities were
used for purposes that were grossly economically inefficient. Second,
market competition for land is a historically contingent cultural
value. Some facts of American cities may be explained by market
efficiency, we need to attend to other values to adequately understand
urban forms. Third, we should not expect that distribution of people
and organizations in a growing city to be globally efficient. The form
of a growing city requires a theory of growth, not a theory of static
equilibrium. 

\subsection*{Distributions of Settlements and Economic Geography}
Sociologists have mainly left geographers to the task of explaining
the size and locations of settlements. The geographers' theories, or
at least the ones that we have attended to, have historically, focused
on two factors,: labor costs and transportation costs of raw material and
finished goods.\footnote{Firey 24}

The geographic distribution of raw materials, transportation network,
manufacturing technology, population, and labor prices combine to form
a geography of costs of operating a firm. Within this economic
geography, certain locations confer advantage to their resident firms
who win out against their poorly positioned
competitors. Advantaged locations would swell while the backwaters
dwindled.\footnote{ullman}

\subsubsection*{Population Ecology}
This economic geography model is a special case of the population
ecology model of Hannan and Freeman.  In economic geography, the
\emph{spatial distribution} of firms is the result of the competition
and selection of firms who have a \emph{locational} advantage over
competitors facing the same market. In the classic formulation of
population ecology, the \emph{distribution of populatoins} of
different types of firms is the result of competition and selection of
firm who have a \emph{strategic} advantage over competitors facing the
same market.\{cite here}. 

Both population ecology and the special case of economic geography are
models of how environments select firms that are somehow better fitted
to the environment. In these models, firms cannot meaningfully change
their environment. This is a very useful simplification and a valid
one for the small restaurants that Hannan and Freeman used as their
first test case.  In economic geography, this simplification may be
less valid, as the geographic concentration of firms is likely to
effect the distribution of people, the costs of labor, and even
transportation and manufacturing technique. We'll pick up this point
again when in the section on mutualistic interactions.

\subsection*{Distributions within Settlements}
Within cities, the transportation costs also create an economic
geography. Across a city, the distribution of people and organizations
interacts with the transportation networks to produce areas of maximum
accessibility. The most accessible place should be dominated by the
type of users who benefit most from accessibility relative to their
unit costs of land.\footnote{Alonso} The unit costs, in turn, were the
outcome of market competition between firms and residents for the
accessibility.

In a rare point of consensus, the early 20th century urban
sociologists agreed that commercial firms most benefited from
acccessibility, so should dominate the area of maximum
accessibility.\footnote{burgess, chauncy and ullman}. Even Firey
allowed that it was economically rational for large department stores
to dominate downtowns.

The central business district, the site of global maximum
accessibility, was the largest nucleus of firms it was not the only
one. There were many locally maximal accessible locations, typically
at the intersection of major streets or transportation lines. Smaller,
local commercial firms should occupy these spaces, forming miniature
central business district.

Manufacturers could also form distinct nuclei of populations. Oriented
to national markets, they would choose locations that minimized their
long distance transport costs and had ample land for large
plants. These locations were often far from the center of population,
but if these large employers were concentrated, they could attract
employees to reside in their proximity.

While sociologists more or less agreed that competition for
accessibility would determine the locations of commerce and industry,
there was substantial disagreement about how people should distribute
themselves in the places left to them.



\todo{note zoning and politics is completely absent from these
  accounts}




Alonso's model, he does not claim that any particular type of land users
will dominate the most accessible parts of the city, but early urban
sociologists all agreed that commercial uses should predominate. 

Chauncey and Ullman specify Alonso's model by arguing that the most
accessible points should be dominated by commercial and retail firms
who benefit most from being accessible by the metropolitan market and
who can outbid other uses. That 


who can pay. In his
schematic model, businesses, of an unspecified variety, benefit more
from being accessible to customers than residential users, so they
should 


commercial firms, since they benefit most from being easily accessible
by customers and 
commerce, since commercial firms benefit m



\todo{touch on macpherson's community matrix}





As such, these models should be used when
firms have little effect upon their environments. 

First, firms have little effect upon their environment. Within
economic geography, the locations of firms does not lead to a
significant redistribution of population or changing labor costs. For
population ecology, firms compete for share of a market but have no
real effect on total market size. In these models, environments cause
populations, populations do not cause environments.

Second, firms interact 



The simplifying assumptions of economic geography purposely omitted
many additional factors. We'll be most interested in two. First, firms,
even competing firms, can benefit from the presence of other firms,
and  


even of the same type, cFor our purposes, the most salient are that
the model only includes competitive interaction between firms, and
that model does 
it is a model of pure competition between firms within a regional or
national market, without any local interactions between
firms. Assuming equivalent locational factors, a firm is equally
affected if a competitor is down the street or in the next state. Even
less, could the firm benefit from 




people and organizations are distributed among settlements if
those theories can be applied to 
distribution of peoples and organizations among settlements has
typically only interested urban 
The study of the allocation of people among settlements has largely
been an exercise of 
The allocations of people and organizations among settlements has
largely been


Based on their primary economic function, Chauncey Harris and Edward
Ullman theorized a


(firey is the one that distinguishes between theoretica modes and
descriptive)



This is actually something else (this probably goes in the global
cities section)
Within a wide
metropolitan areas, the most highest administrative corporate
functions and most specialized services would, efficiently, centralize
in central city, serving clients and subsidary branh operations
hundreds of miles away.







Within settlements, all people and organizations
preferred 



In 



 been both used in both explicit theories of
urban forms and an often implicit mode of analysis. Explicit theories
first identify a spatial distribution of resources that were fixed or
slow-changing, and describe how people and organizations will
distribute themselves to optimizes access to those resources. In the
implicit mode, the analyst makes no special attempt to distinguish a
fixed, or slow changing environment from the relatively quickly
changing people or organizations of interest. Instead the analyst
makes a ceteris parebus assumption that the distribution of resources
are fixed and shows how actors seem to be respond to assumed fixed resources.



\subsection*{Subdominance}

\subsection*{Population Ecology}

\subsection*{Creative Class}

Some of the first theories of The first theories of 

The earliest explanations of 

The earliest explanations for 


Among biological species, when species A benefits from the presence of
species B, but species B is not affected by actor A, they have a
"commensalistic" interaction. Examples are species which eat the
carcasses or refuse of another species. In other words, one species is
a resource for the the other without being much affected by it. 

Resource dependence is the most common form of explanation for the
spatial distribution of


