\section*{Competition} 
In the late 19th century and early 20th century, American urban
sociologists grappled with the unprecedented size and rapid growth of
the industrial city. Their main explanation for these astonishing
social facts was that aggregations of people and organizations into
dense settlements and the locations of those settlements across the
land, were economically efficient. The distribution of people and
organizations among settlements tended to minimize the total costs of
transportation for the raw materials and finished goods of industrial
manufacturing. Within settlements, the central places should be
dominated by uses that most benefited from access the city or be
accessible by the city.

In these efficiency arguments, economic competition between firms was
the ultimate mechanism. Firms in places with cheap access to inputs
and markets survived and their less well placed competitors did
not. Populations followed jobs and settlements waxed or waned on the
fortunes of its firms competing in a national market.  An efficient
allocation of people and organizations across and within settlements
was the equilibrium state outcome of that 

Contemporary critics attacked the competitive models on three
fronts. First, empirically, many districts of American cities were
used for purposes that were grossly economically inefficient. Second,
market competition for land is a historically contingent cultural
value. Some facts of American cities may be explained by market
efficiency, we need to attend to other values to adequately understand
urban forms. Third, we should not expect that distribution of people
and organizations in a growing city to be globally efficient. The form
of a growing city requires a theory of growth, not a theory of static
equilibrium. 

\subsection*{Distributions of Settlements}
Based on their primary economic function, Chauncey Harris and Edward
Ullman 




This is actually something else (this probably goes in the global
cities section)
Within a wide
metropolitan areas, the most highest administrative corporate
functions and most specialized services would, efficiently, centralize
in central city, serving clients and subsidary branh operations
hundreds of miles away.







Within settlements, all people and organizations
preferred 



In 



 been both used in both explicit theories of
urban forms and an often implicit mode of analysis. Explicit theories
first identify a spatial distribution of resources that were fixed or
slow-changing, and describe how people and organizations will
distribute themselves to optimizes access to those resources. In the
implicit mode, the analyst makes no special attempt to distinguish a
fixed, or slow changing environment from the relatively quickly
changing people or organizations of interest. Instead the analyst
makes a ceteris parebus assumption that the distribution of resources
are fixed and shows how actors seem to be respond to assumed fixed resources.



\subsection*{Subdominance}

\subsection*{Population Ecology}

\subsection*{Creative Class}

Some of the first theories of The first theories of 

The earliest explanations of 

The earliest explanations for 


Among biological species, when species A benefits from the presence of
species B, but species B is not affected by actor A, they have a
"commensalistic" interaction. Examples are species which eat the
carcasses or refuse of another species. In other words, one species is
a resource for the the other without being much affected by it. 

Resource dependence is the most common form of explanation for the
spatial distribution of



