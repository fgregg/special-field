\section{Introduction}
Urban sociologists have used, mainly, three ecological principles to
explain why things are where they are: resource dependence, competition,
and mutualism.

The first principles is that peoples and organizations need some
resources which are spatially fixed and rare. In order to access those
resources, people and organizations will locate near them. For
example, mining towns appear near mineral deposits and high tech
businesses form in cities with a highly educated workforce.

Where there are rare resources, there is likely to be competition and
this is the second common principle. The presence of particular
peoples or organizations is not just due to the presence of resources,
but the succesful and exlusionary competition for those resources.
For example, almost all retail businesses would like to locate in the
place of maximum accessibility, but only certain businesses can afford
the rent.

Finally, the mutualism emphasizes how people and
organizations form the environments for one another. Jeweler's locate
near each other because the benefits for of forming a district that
attracts customers who want to search and compare goods outweighs the
losses competition. Industries cluster geographically because
productions requires the coordinated inputs of different
firms. Entertainment districts, and other scenes, emerge because
certain businesses attract the types of people who want to enact a
certain lifestyle and those people form the consumer base for certain types
of businesses to prosper.

Within the discipline of ecology, interactions between species are
often characterized by the net effect of the presence of one species
upon another. For one species, say wolves, the presence of plentiful
rabbits increases the number of wolves, plentiful hunters decreases
wolves, and plentiful morning glories has no effect. A pair of species
can have one of six types of interactions show in Table~\ref{tab:ecological}

\begin{table}[h]
\centering
\begin{tabular}{lcccl}
  & +         & 0                        & -           &  \\
+ & Mutualism & Resource Dependence      & Antagonism  &  \\
0 &           & Neutralism               & Amensalism  &  \\
- &           &                          & Competition & 
\end{tabular}
\caption{Ecological Interactions}
\label{tab:ecological}
\end{table}

In this paper, we will explore how sociologists have used these types
of interactions to understand the ``spatial and temporal distribution
of social and cultural data'', \footnote{Gettys} particular within and
among cites.  We will focus on resource dependence, competition, and
mutualism.  The remaining three types of interactions have been little
used by urban sociologists, but I will touch on a few examples.

Our discipline has mainly explained cities and urban form through
resource dependence and competition for scarce resources. However,
since the 1990s, mutualism has emerged as the major 
explanation for the rennaissaince of American central cities
cities. However, the full logic of this turn has not been developed,
and it is here where I will spend the bulk of the paper.

