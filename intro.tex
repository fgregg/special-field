\section{Introduction}
Every urban process depends upon the size and composition of a city's
human and organization population. Most processes also depend upon the
spatial organization of those populations. Politics, culture,
economics, and social order cannot be reduced to demography and
population geography, but any analysis of the inter-city differences in
politics, culture, economics, and social order starts with asking who
composes the cities and how they are arrayed.

Urban sociology has, therefore, always been interested in two
questions. First, why are there cities? That is, why are people and
organizations concentrated geographically? Second, within an urban
area, why are people and organizations where they are?

For contemporary cities, sociologists have mainly answered these
questions through theories of interaction among firms; among people;
or among firms and people both. In these theories, large scale
patterns emerge, unintentionally, from the interaction of many
individual firms or people pursuing their own, individual aims. 

While sociologists have not always used these labels, the theories of
interaction have been one of three types: competition, antagonism, or
mutualism. Early American sociologists explained the growth of the
industrial city as the result of the market competition between firms,
in a national market, and the advantages of certain locations in that
competition. Similarly, within cities, the most accessible locations
should come to be dominated by commercial firms that could and would
outbid other types of uses. 

Within cities, the racial, ethnic, and occupational residential
segregation were explained, largely, with the theories of
antagonism. A higher status group could not stably abide in the
presence of lower status group. When low status groups entered into an
area of higher status residence, the neighborhood should transition
rapidly and until status segregation was re-established.

After World War II, technological and economic changes in
transportation, manufacturing, and consumption destroyed the economic
advantages of American industrial cities. In line with theory, the
dense populations of firms and people did disperse. But, by the 1970s,
commercial districts of large American cities were booming and that
high status groups were beginning to live in cities alongside the
lowest status groups in our society. 

Urban sociology needed new theories for urban agglomeration. The new
theories have largely been theories of mutualism. Certain areas, like
Silicon Valley or Wall Street, possess a particular population of
firms and workers that make it easier for complementary firms and
workers to survive and thrive. Similarly, other areas like Orange
County or Portland, Oregon possess a mix of type of people and retail
venues that facilitated the expression and development of a subculture
that attracts more, complementary people and firms.

In this paper, I will review all three types of theories of
interaction and explore how they answer the questions of why cities
and how are cities internally organized. But, before we begin, let me
explain the type of explanations I will be omitting. 

The existence, populations, and geography of cities are caused by of
state action. Washington, DC exists because of an act of
Congress. Racial residential segregation was maintained for decades
because of the state-backed restrictive covenants. Urban renewal was
funded and enforced by the federal government. Redlining was federal
policy that contributed to racial segregation and urban
disinvestment. The size and extension of municipal borders are
regulated by state legislatures. These are non-interactional
explanations: large scale patterns are the direct result of the action
of just a few powerful, large scale actors, not the emergent result of
many actors taking local action. I will not be attending to these
types of non-interactional explanations in this paper.





The first question of urban sociology is why are people and
organization where they are. 


At the national and international level,
why do people and organizations concentrate into urban
agglomerations? Within urban areas, why are different types of people
and organizations arrayed as they are?

These questions are fundamental because every urban process is largely
determined by the size, composition, and spatial arrangement of human
and organizational populations. As we'll see a city's politics,
culture, and social order can affect its future population and spatial
organization, but


These questions are not, however,
logically prior to 

While most works of urban sociology do not take these questions as
their primary focus, the human and organization population and their
spatial organization affect every aspect of city life from politics to
culture; economic growth to intergenerational poverty. If not in the
foreground, the who and where of urban areas is always in the background.

In this paper 





Urban sociologists have used, mainly, three ecological principles to
explain why things are where they are: resource dependence, competition,
and mutualism.

The first principles is that peoples and organizations need some
resources which are spatially fixed and rare. In order to access those
resources, people and organizations will locate near them. For
example, mining towns appear near mineral deposits and high tech
businesses form in cities with a highly educated workforce.

Where there are rare resources, there is likely to be competition and
this is the second common principle. The presence of particular
peoples or organizations is not just due to the presence of resources,
but the succesful and exlusionary competition for those resources.
For example, almost all retail businesses would like to locate in the
place of maximum accessibility, but only certain businesses can afford
the rent.

Finally, the mutualism emphasizes how people and
organizations form the environments for one another. Jeweler's locate
near each other because the benefits for of forming a district that
attracts customers who want to search and compare goods outweighs the
losses competition. Industries cluster geographically because
productions requires the coordinated inputs of different
firms. Entertainment districts, and other scenes, emerge because
certain businesses attract the types of people who want to enact a
certain lifestyle and those people form the consumer base for certain types
of businesses to prosper.

Within the discipline of ecology, interactions between species are
often characterized by the net effect of the presence of one species
upon another. For one species, say wolves, the presence of plentiful
rabbits increases the number of wolves, plentiful hunters decreases
wolves, and plentiful morning glories has no effect. A pair of species
can have one of six types of interactions show in Table~\ref{tab:ecological}

\begin{table}[h]
\centering
\begin{tabular}{lcccl}
  & +         & 0                        & -           &  \\
+ & Mutualism & Resource Dependence      & Antagonism  &  \\
0 &           & Neutralism               & Amensalism  &  \\
- &           &                          & Competition & 
\end{tabular}
\caption{Ecological Interactions}
\label{tab:ecological}
\end{table}

In this paper, we will explore how sociologists have used these types
of interactions to understand the ``spatial and temporal distribution
of social and cultural data'', \footnote{Gettys} particular within and
among cites.  We will focus on resource dependence, competition, and
mutualism.  The remaining three types of interactions have been little
used by urban sociologists, but I will touch on a few examples.

Our discipline has mainly explained cities and urban form through
resource dependence and competition for scarce resources. However,
since the 1990s, mutualism has emerged as the major 
explanation for the rennaissaince of American central cities
cities. However, the full logic of this turn has not been developed,
and it is here where I will spend the bulk of the paper.

