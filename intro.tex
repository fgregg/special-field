\documentclass[12pt,draft,letter]{article}

\begin{document}


\section{Introduction}
Sociologists have had two broad approaches to explaining why things
are where they are: evolutionary and co-evolutionary.

The evolutionary approaches attempts to explain the spatial
distribution of people, organizations, and social practices as due to
relatively fixed environmental factors. For example, the mining towns
appear near mineral deposit; high tech businesses form in cities with
a creative class; and department stores will compete for the area of
maximum accessibility and traffic. Within the evolutionary approach,
the main form of interaction between populations is competition for
scarce environmental resources.

The co-evolutionary approach emphasizes how people and organizations
form the environments for one another. The main form of interaction
are mutualistic. Jeweler's locate near each other because the benefits
for of forming a district that attracts customers who want to search
and compare goods outweighs the losses competition. Industries cluster
geographically because productions requires the coordinated inputs of
different firms. Entertainment districts, and other scenes, emerge
because certain businesses attract the types of people who want to
enact a certain lifestyle and those people form the market for certain
types of businesses to prosper.

In this paper, I will be reviewing how sociologists and other
scientists have used these two approaches to understand the ``spatial
and temporal distribution of social and cultural
data'', \footnote{Gettys} particular within and among cites. While
both approaches can be found throughout the sociological studies of
cities, the evolutionary approach dominated throughout the 1950s, and
the co-evolutionary approach only emerged strongly in the 1990s. 

In the early period, sociologists were grappling with the unprecedented
size and rapid growth of the industrial city, and the main answer was
that moving people, goods, or information around was expensive and
that the concentrated distribution of people and organizations of the
industrial city was efficient.

Cities that enjoyed easy transportation and communication with the
national or regional market would prosper and those that did not would
wither. Within cities, commercial uses would dominate the place of
maximum accessibility; industrial uses would locate in areas of
maximum convenience for shipping of their inputs and
outputs. Residents sorted themselves in the places that were left,
first by what they could prefer, and then on other dimensions, often
unspecified.

Contemporary critics recognized that distributions of land uses on the
basis of market efficiency required a culture that valued of market
efficiency and valued it over other cultural ideas. Perhaps impossible
to recognize at the time, the market efficiency of the industrial city
also depended upon a particular constellation of manufacturing
practice, long and short distance transit costs, communication
technology, and labor conditions. 

While this is not the place to review the emergence of Post-Fordist
industrial production, suffice to say that by the 1970s it was no
longer economical for manufacturers to locate massive factories within
central cities and that mere propinquity to markets no longer provided
much competitive advantage.

While manufacturers and department stores abandoned central cities,
the largest American cities started added the center of some American
cities started to grow, first in office space in the center of the
city and then with residential upgrading for the new, young workers
who worked in the offices. 

Sociologists needed a new explanation for this urban rennaissance and
have articulated two co-evolutionary arguments. First, there emerged
new types of highly paid work that requires in-person coordination
between firms and is thereby facilitated by propinquity, and
second. The high paid workers who do the type of work like living in
cities for lifestyle reasons.

In both explanations, there are network effects. The more firms there
are that specialize in the in the complicated production of these
goods, the more productive the location, which will allow for more
firms and more specialization. Similarly, the more people live in a
city and enjoy a particular consumption pattern, the more
organizations will emerge to cater to that pattern, making the
location more attractive for people with that lifestyle.




            Space                    Place
Production  Technology Determinism   Coevolutionary   
Reaction    Time Distance            Population Ecology







Space vs place distiction (alfred)


Broad strokes, complete, let usnow 



\end{document}
