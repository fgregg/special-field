\documentclass[12pt,draft,letter]{article}

\begin{document}


\section{Introduction}
Sociologists have always been concerned with why things are where they
are. This project used to be called ecology, of some variety: ``social
ecology'', ``human ecology'', or ``urban ecology.'' These days,
`ecology' is unfashionable, but the work continues in the study of
gentrification; economic clusters; social, cultural, and economic
geography; globalization; scenes; and creative classes.

Within this project, there have been two broad approaches that I'll
call evolutionary and co-evolutionary. The evolutionary approaches
attempts to explain the spatial distribution of people, organizations,
and social practices as due to relatively fixed environmental
factors. For example, the mining towns appear near mineral deposit;
high tech businesses form in cities with a creative class; and
department stores will compete for the area of maximum accessibility
and traffic. Within the evolutionary approach, the main form of
interaction between populations is competition for scarce
environmental resources.

In the co-evolutionary approach, scholars emphasize how people and
organizations form the environments for one another. The main form of
interaction are mutualistic. Jeweler



The various sociological ecologies were borrowed different ideas from the
discipline of ecology: succession, dominance, selective adaptation,
and energy flow. 



``[that] center[ed] its attention upon the description, measurement,
analysis and explanation of the spatial and temporal distribution of
social and cultural data''



From it's beginning, sociology addressed itself to urbanization 


Within sociology, ecology has meant a great many things: the
succession of typical uses of urban space; a holistic vision of the
co-evolution between social form and technologically mediated access
to ``natural'' resources; selective adaptation of populations of
organizations to their environments. In this essay, we hew close to an
early definition:



\end{document}
