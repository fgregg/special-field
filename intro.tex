\section{Introduction}
Every urban process depends upon the size and composition of a city's
human and organizational population. While politics, culture,
economics, and social order cannot be reduced to demography, any
analysis of the inter-city differences in these processes must start
with asking who composes the cities and how they are arrayed.

Urban sociology has always been interested in two
questions. First, why are there cities? That is, why are people and
organizations concentrated geographically? Second, within an urban
area, why are people and organizations where they are?

Sociologists have mainly answered these questions through theories of
interaction among firms; among people; or among firms and people
both. In these theories, large scale patterns emerge, unintentionally,
from the interaction of many individual firms or people pursuing their
own, individual aims.  While sociologists have not always used these
labels, the theories of interaction have been one of three types:
competition, antagonism, or mutualism.

Early American sociologists explained the growth of the
industrial city as the result of the economic competition between firms,
in a national market, and the advantages of certain locations in that
competition. Similarly, within cities, the most accessible locations
should come to be dominated by commercial firms that could and would
outbid other types of uses. 

Within cities, the racial, ethnic, and occupational residential
segregation were explained, largely, with the theories of
antagonism. A higher status group could not stably abide in the
presence of lower status group. When low status groups entered into an
area of higher status residence, the neighborhood should transition
rapidly and until status segregation was re-established.

After World War II, technological and economic changes in
transportation, manufacturing, and consumption destroyed the previous
economic advantages of American industrial cities. In line with
theory, the dense populations of firms and people did disperse. But,
by the 1970s, commercial districts of large American cities were
booming and that high status groups were beginning to live in cities
alongside the lowest status groups in our society.

Urban sociology needed new theories for urban agglomeration. These
have largely been theories of mutualism. Certain areas, like Silicon
Valley or Wall Street, possess a particular population of firms and
workers that make it easier for complementary firms and workers to
survive and thrive. Similarly, other areas like Orange County,
California or Portland, Oregon possess a mix of type of people and
retail venues that facilitated the expression and development of a
subculture that attracts more, complementary people and firms.

In this paper, I will review all three types of theories of
interaction and explore how they answer the questions of why cities
exist and how cities internally organized. 

Schematically, we can organize these theories of interaction into a
two by two table based on how the inter-actors affect each other
(Table~\ref{tab:ecological}). In a competitive interaction, the
presence of group A inhibits the presence of group B and the presence
of group B inhibits group A. In an antagonistic interaction, the
presence of group A inhibits group B, but the presence of B helps
group A. In a mutualistic interaction, the presence of
group A helps group B and the presence of group B helps group A.

\begin{table}[h]
\centering
\begin{tabular}{ccc}
  & +          & -           \\
+ & Mutualism  & Antagonism  \\
- &            & Competition  
\end{tabular}
\caption{Ecological Interactions}
\label{tab:ecological}
\end{table}

We'll discuss all three of these types of interaction in turn. Of
these types of interaction, mutualism is the newest and least worked
out in urban sociology. I will spend particular time on the logic of
mutualism and the role of coherence in theories of mutualism.

Before we begin, though, let me note a class of explanations for
cities that I will be omitting.

Many aspects of the population and geography of cities do not emerge
from the interaction of many local actors, but are the direct
consequence of an agents capable of affecting conditions at a city,
state, or national scale. In the United States, these agents have
mostly been local, state, or federal government. An act of Congress
decreed the existence of Washington, DC. State-backed restrictive
convenants helped maintain racial segregation for decades. State and
federal power funded and enforced urban renewal. Redlining was federal
policy that contributed to racial segregation and urban
disinvestment. State legislatures regulate the size and extension of
municipal borders. City governments largely determine what types of
people and firms can exist where through zoning law and building
permits.

Sociologists certainly recognize the power of state actors in shaping
the history of any particular American city, but have preferred
interactional theories for urban agglomeration and within-city
patterning of landuse.\footnote{mann sources of power} While a theory
of the state driven urban agglomeration would be fascinating to
develop, in this paper we follow the main current, and focus on
interactional theories of the city.


