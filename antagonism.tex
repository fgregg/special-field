\section*{Where do the People Go}
In the classic models, the distribution of natural resources, market
populations, and transportation networks form an geography of better
or worse places to locate business functions. The best locations will
be occupied by businesses, which will, in turn produce a geography of
commute times for residents.

In the early industrial period, when walking was the main mode of
transportation, this geography hemmed growing population to within a
few miles of their places of employment. Within a century, though,
rapid improvements in short distance transportation had relaxed the
geographic connection between work and homes.\footnote{Hawley 90}. By
the 1920s, the major American theorists of urban form recognized that
the least accessible locations, the edges of the city, contained the
most attractive residential locations.\footnote(burgess, 53, hoyt}

Unlike firms, households did not seem to be choosing locations to
minimize the cost of land and transportation. A different type of
theory was necessary to explain residential choice, and for the most
part, early 20th century theorists put forward theories of amensalism
between different types of uses and people. Ecologically, amensalism 
is when the presence of group A harms group B but the presence of
group B does not effect group A.

In Ernest Burgess's famous concentric zone theory, in an expanding
city, the central business district will expand into a zone of working
class residences. Through population growth or displacement by the
expanding central business district, working class residents encroach 
into a zone of wealthier residences, who move further away to avoid
their new neighbors.\footnote{burgess 50}

Homer Hoyt follows the same line when he observed that a growing
population of immigrants and non-whites in Chicago promoted the
expansion of the city because these groups ``forced or stimulated the
old American stock to see new neighborhoods and has caused them to
migrate from their old homes.''\footnote{hoyt, 317}

Does hoyt have another explanation for why rich and poor are
segregated besides antipaty of the rich for the poor?

Hoyt, of course, had additional theory of why 

\todo{The concept of filtering actually seems more antagonistic}
\todo{how do we think of shelling}







\local transit line as factor in sectoral theory

H
