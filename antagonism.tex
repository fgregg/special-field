\section*{Antagonism}
While the economic competition of the industrial age explained why
firms concentrated in cities, economic competition could not explain
much about why people lived where they did within a city. A simple
theory of economic competition would predict that the locations that
minimized the total travel time would be dominated by the
wealthiest residents. Instead, the neighborhoods closest to the center
of the cities or clusters of employment were often occupied by the
poorest citizens and richer residents seemed to have positive preference
for locations further away from the city.

Some theorists argued that residential choice could be explained
if we better understood what people wanted (clean air and
water, a room of one's own, access to recreation, or sentimental
attachment of historic neighborhood). However, the main line of theory
explained residential patterns as the result of antagonistic
interactions between status groups.

Different status groups should be residentially segregated because
of the antagonism between higher and lower status groups. When lower
status groups entered a higher status area, the higher status group
should quickly remove themselves to a new district, re-establishing
status segregation.

\subsection*{Where do the People Go}
In the classic models of the industrial city, the distribution of
natural resources, market populations, and transportation networks
form an geography of better or worse places to locate business
functions. The best locations will be occupied by businesses, which
will, in turn produce a geography of commute times for residents.

In the early industrial period, when walking was the main mode of
transportation, this geography held growing population to within a
few miles of their places of employment. Within a century, though,
rapid improvements in short distance transportation had relaxed the
geographic tie between work and homes.\cite{hawley_urban_1981}. By
the 1920s, the major American theorists of urban form recognized that
the least accessible locations, the edges of the city, contained the
most attractive residential locations.\cite{park_growth_1984,hoyt_one_1970}


Unlike firms, households in the industrial city did not seem to be
choosing locations to minimize the cost of land and
transportation.\footnote{Although travel time would still have a
  signfiicant influence, as evidenced by the faster development of
  cities along the main transport routes.} A different type of theory
was necessary to explain residential choice. For the most part,
early 20th century theorists put forward theories of antagonism
between different types of uses and people.\footnote{The other main
  alternatives was the attractions of beauty, (clustering on the
  lake)Hoyt, and Alonzo's speculation that utility tradeoff between
  the amount of land and accessibility leads the poor to live in the
  inner city in cramped quarters and the wealthy to live further away
  on spacious lots.} Ecologically, antagonism is when the presence of
group A harms group B and the presence of group B helps group A.

In Ernest Burgess's famous concentric zone theory, in an expanding
city, the central business district will expand into a zone of working
class residences. In turn, through population growth or displacement by the
expanding central business district, working class residents encroach 
into a zone of wealthier residences, who move further away to avoid
poorer neighbors.\cite{park_growth_1984}

Homer Hoyt followed the same line when he observed that a growing
population of immigrants and non-whites in Chicago promoted the
expansion of the city because these groups ``forced or stimulated the
old American stock to see new neighborhoods and has caused them to
migrate from their old homes.''\cite{hoyt_one_1970}

In these classic theories, the focus is on the fleeing populations.
Neither Burgess nor Hoyt much discussed whether the encroaching land
uses or social groups were helped or harmed by the incumbent
populations.

\subsection*{Factorial Ecology}
Factorial ecology carried forward the idea that populations of higher
status groups were harmed by the presence of lower status group, but
also, explicitly, argued that lower status groups were drawn towards
populations of higher status.

In the mid-centry, factorial ecologists demonstrated that status
segregation was the rule for large cities across the world. While,
early factorial ecology was more focused in describing the
socio-spatial organization of cities then explaining them,\cite{shevky_social_1955}
later analysts emphasized antagonistic mechanisms: ``residential
relocation may be seen as strategies for minimizing the social
distance between the individual and populations which he desires to
emulate and for maximizing the that from which he wishes to leave
behind.''\cite{timms_urban_1975}

Since the mid-century, the simple status order that underpinned both
the this antagonistic theories have broken down.  Against expectation,
in the latter part of the centry, high status groups started returning
to the city and displacing the poor, even without heroic efforts of
urban renewal.

Ironically, urban sociologists have also identified the return of the
rich as an antagonistic process. Gentrification is often stereotyped
as a two stage process. First, artists and bohemians move into poor or
depressed, nonresidential area because of cheap rent and, perhaps, a
taste for urban grit. They are followed by high income young
professionals who bid up the cost of land forcing both artists and the
original, incumbent population out. The population of artists settle
on another area of poor rent and the cycle can continue.\cite{ley_new_1997}

\subsection*{Spatial Dynamics of Antagonistic Interactions}
Antagonistic interactions leads to a dynamic of perpetual
desegregation and resegregation. In the classic status antagonism, low
status groups are drawn to higher status areas, causing high
status groups to exit the area and resegregate elsewhere, only to be
pursued by low status groups again.

If the antagonistic groups are free to move across unbounded space,
then antagonistic interactions should lead to a kind of wave
front. The highest status groups should perpetually be pushing out
from the edge of the city, soon pursued by their low-status cousins.

If the groups are confined in a bounded space, then that wave will
eventually hit a limit, and the high status group will have to return
to a previous location. If the high status group has successfully led
the low status groups to the edge of the city, then the high status
group may return to the center.

If status groups are antagonistic interaction, then neighborhood
stability is a puzzle. Residents of high status areas face a
persistent collective action problem. They would like their
neighborhood to maintain it's high status character, but that requires
concerted action to keep out lower status groups. Richard Taub
describes both this problem and potential solutions in \emph{Pathways
  to Neighborhood Change}, but successful coordination seems difficult
and rare.\cite{taub_paths_1987}

The stability of low status areas is also be puzzling. Why doesn't the
neighborhood empty out once the higher status groups have moved
further away. Of course, some neighborhoods do, leaving ``truly
disadvantaged'' residents stranded in nearly empty
neighborhoods.\cite{wilson_truly_1990} However, many areas have been dense,
low-status areas for generations even without substantial immigration.
