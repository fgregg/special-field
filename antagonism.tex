\section*{Where do the People Go}
In the classic models, the distribution of natural resources, market
populations, and transportation networks form an geography of better
or worse places to locate business functions. The best locations will
be occupied by businesses, which will, in turn produce a geography of
commute times for residents.

In the early industrial period, when walking was the main mode of
transportation, this geography hemmed growing population to within a
few miles of their places of employment. Within a century, though,
rapid improvements in short distance transportation had relaxed the
geographic connection between work and homes.\footnote{Hawley 90}. By
the 1920s, the major American theorists of urban form recognized that
the least accessible locations, the edges of the city, contained the
most attractive residential locations.\footnote(burgess, 53,
hoyt}. 

Unlike firms, households did not seem to be choosing locations to
minimize the cost of land and transportation.\footnote{Although travel
  time would still have a signfiicant influence, as evidenced by the 
  faster development of cities along the main transport routes.} A
different type of theory was necessary to explain residential choice,
and for the most part, early 20th century theorists put forward
theories of amensalism between different types of uses and
people.\footnote{The other main alternatives was the attractions of
  beauty, (clustering on the lake)Hoyt, and Alonzo's speculation that
  utility tradeoff between the amount of land and accessibility leads
  the poor to live in the inner city in cramped quarters and the
  wealthy to live further away on spacious lots.} Ecologically,
amensalism is when the presence of group A harms group B but the
presence of group B does not affect A.

In Ernest Burgess's famous concentric zone theory, in an expanding
city, the central business district will expand into a zone of working
class residences. Through population growth or displacement by the
expanding central business district, working class residents encroach 
into a zone of wealthier residences, who move further away to avoid
poorer neighbors.\footnote{burgess 50}

Homer Hoyt follows the same line when he observed that a growing
population of immigrants and non-whites in Chicago promoted the
expansion of the city because these groups ``forced or stimulated the
old American stock to see new neighborhoods and has caused them to
migrate from their old homes.''\footnote{hoyt, 317}

In these classic theories, the focus is on the fleeing populations.
Neither Burgess nor Hoyt much discussed whether the encroaching land
uses or social groups were helped or harmed by the incumbent
populations.

Assuming stable populations, amensalistic interactions between
different status groups should lead to stable residential
segregation. However, when a lower status population group should move
into a high status area, they should rapidly and displace thee higher
status groups.

This dynamic captures much of the settlement patterns of growing,
industrial American cities through the mid century, particularly the
white flight and suburbanization of post-war period. When the rapid
influx of poor immigrants stopped with the onset of
deindustrialization, the poorest, lowest status groups remained in
place, in the center city. (juliius wilson).

\subsection{Antagonism}
The later theories of factorial ecology extended the status amensalism
of Burgess and Hoyt into antagonism. Factorial ecology carried forward
the idea that populations of higher status groups were harmed by the
presence of lower status group, but also, explicitly, argued that
lower status groups were drawn towards populations of higher status.

In the mid-centry, factorial ecologists demonstrated that status
segregation was the rule for large cities across the world. While,
early factorial ecology was more focused in describing the
socio-spatial organization of cities then explaining them (shevsky),
later analysts emphasized antagonistic mechanisms: ``residential
relocation may be seen as strategies for minimizing the social
distance between the individual and populations which he desires to
emulate and for maximizing the that from which he wishes to leave
behind.'' timms 98

Assuming a stable populations and status differences, this antagonism
should lead to constant desegregation and resegregation of status as
the rich attempt to escape their poor cousins but are chased by the
strivers among the poor. This does seem to be happening.

Since the mid-century, the simple status order that underpinned
both the earlier amensalistic and antagonistic theories have broken down.
Against expectation, in the latter part of the centry, high status
groups started returning to the city and displacing the poor, without
heroic efforts of ``urban renewal.''

Ironically, this has also theorized as an antagonistic
process. Gentrification is often stereotyped as a two stage
process. First, artists and bohemians move into poor or depressed,
nonresidential area because of cheap rent and, perhaps, a taste for
urban grit. They are followed by high income young professionals who
bid up the cost of land forcing both artists and the original,
incumbent population out. The population of artists settle on another
area of poor rent and the cycle can continue.{ley}

In our last section on mutualism, we will return to this breakdown of
status order and how that is reshaping the city


\todo{how do we think of shelling; schelling, put him in with mutualism }








H
