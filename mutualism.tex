\section*{Mutualism}
If the survival of firms depends on the place based costs of
production and transportation, then changes in the geography of those
costs will change the distribution of those types of firms. In the
early industrial period, transportation costs dominated other
locational factors, leading to the concentration of industry in key
nodes of the transportation. After the second world war, rising wages,
trucking, containerization, and new manufacturing techniques
undermined earlier advantages of the industrial city. Total
transportation costs became a secondary or tertiary consideration, and
the manufacturing firms first left the city often, later, the country.

For the theorists of the industrial city, the locational efficiencies
for firms were the fundamental reason for the city and its
growth. When the city was no longer efficient for longer efficient,
the city should dissipate. Indeed, it seemed to be doing just that. 

However, by the 1980s, the growth of the corporate service sector and
the return to the city of high status groups confounded expectations
and demanded new explanations of urban form. Two main theories have
developed. First, that some industries require in-person coordination
between firms in order to thrive, so firms location near one
another. Second, some people live in cities in order to enact
lifestyles that are particularly available in cities. 

As we will see, these are both theories of mutualism. Mutualism is
where the presence of group A benefits group B and the presence of
group B benefits group A. 

\subsection*{Clustering and Craft}
Within urban sociology, the most prominent proponents of firm-level
mutualism have been Saskia Sassen and Michael Porter.

Saskia Sassen's work on global cities makes two basic moves. First,
she extends the Donald Bogue's concept of metropolitan dominance to a
globalized economy. As the size and scale of international production
expands, she argues, there is a complementary increase in centralized
administrative functions. Second, and going beyond Bogue, she argues
that those administrative functions should be concentrated in certain
sites because of these administrative functions are the joint product
of multiple firms, which requires intensive coordination facilitated
by proximity: ``At the current stage of technical development, having
immediate and simultaneous access to the pertinent experts is still
the most effective way to operate, especially when dealing with a
highly complex product.''\footnote{95}

Sassens argument about how proximity facilitates the production of
complicated and innovative services is a specific case of generic
argument that Michael Porter makes about ``industry clusters.''
Geographic clustering of an industry can, he argues, provide a number
of benefits for international competition. A concentration of
competitors, suppliers and customers can lead to specialization and
efficiency. Suppliers and competitors can are better able to
communicate and coordinate the production of new products and
services. Proximity lets industry participants better survey each
other and understand the overall state of the industry, allowing for
firms to more quickly react to changes and identify
opportunities.\footnote{157}

The economic engines of post-industrial cities are largely
administered like pre-industrial craft work. The work is variable,
creative, unique, and often customized for the client. As Stinchcombe
predicted for this type of work, the production is largely administed
along craft lines. \footnote{Stinchome} That is, the work is
accomplished by a network of specialist firms, that coordinate
extensively but preserve their own jurisdictions of work. That some of
these firms are enormous investment banks, consulting companies,
medical device companies, or automotive suppliers does not materially
change the craft-like nature of the administration of the work.

Given current communication technology, cities are good for craft-like
work. If the work can be reorganized for industrial production, the
city loses its advantage.

\subsection{Lifestyle}
Like earlier theorists, Sassen and Porter see firms as the prime
movers for the cities. However, other, new theories of cities have
elevated the resident as the driver of urban concentration and
growth. Early work in this line saw a new types of status groups
choosing city life as a form of consumption. Terry Clark and Dan
Silver's recent work goes beyond this to contemplate the co-production
of types of people and types of places. This is a mutualism between
people, between firms, and between people and firms.

Richard Florida's work has been typical of the main lifestyle
arguments. In his model, the workers the human capital necessary for
post-industrial production tend to like living places that are
diverse, tolerant, and creative. These workers choose a place to live
based upon those preferences, not primarily, upon where they can get a
job. High value firms and innovation and economic growth follow the
people.\footnote{richard florida, entertainmenet machine, sassen,
  berry}

This is not just a reversal of the firm-centric models of the
industrial city, it's is also incompatible of with the earlier
status-based theories of residential selection. Those status based
theories assumed that there was a relatively stable status order based
on race, ethnicity, and occupation (and maybe caste, tribe, or
religion in other countries). People might not express their
residential preferences in terms of status, but their choices
effectively revealed a status preference. The lifestyle arguments
contradict this, arguing that an important group of people want to
live where it's fun, exciting, diverse. If the city or neighborhood is
gritty--poor and rundown, then that can be good too.

The lifestyle argument does not reject the theory that people choose
where to live in order to minimize the social distance between
themselves and the population they wish to emulate; it argues that
status groups reduced to a linear order.

However, a theory that people choose where to live based upon
lifestyle preferences is not a theory of why people cluster in
cities. We need an additional link between geographic concentration
and lifestyle. For lifestyle to explain why people move to cities,
there must be an explanation of why cities are especially amenable to
the crucial lifestyle. 


Claude Fischer's subcultural theory provides one such account.
In this theory, cities stimulate expressive unconventionality because
of they stimulate subcultures. Fisher enumerates many mechanisms for
why cities generate and strengthen subculture and why a diversity of
subcultures, but they are first or second order effects size or
growth. In his account, unconventionality is an effect of
agglomeration not a cause.\todo{expand on this}

In their recent work on ``scenes'', Terry Clark and Daniel Silver
married a theories of subcultures and lifestyle based residential
choice to form the first, explicit theory of urban agglomeration
driven by lifestyle. In their account, all people in advanced
economies are increasingly choosing a place to live in order to choose
a scene where they can express their aesthetics. A place's scene is
made by the current residents, both directly and indirectly through
the retail institutions that serve residents' lifestyles. When more of the
right type of people are in a scene, there are more opportunities for
expressive coaction and more institutions serving as venues for
expressive action.

\subsection*{Mutualism and Space}
Mutualistic interactions 

\subsection*{Coherence}




