\section*{Mutualism}
If the survival of firms depends on the place based costs of
production and transportation, then changes in the geography of those
costs will change the distribution of those types of firms. In the
early industrial period, transportation costs dominated other
locational factors, leading to the concentration of industry in key
nodes of the transportation. After the second world war, rising wages,
trucking, containerization, and new manufacturing techniques
undermined earlier advantages of the industrial city. Total
transportation costs became a secondary or tertiary consideration, and
the manufacturing firms first left the city often, later, the country.

For the theorists of the industrial city, the locational efficiencies
for firms were the fundamental reason for the city and its
growth. When the city was no longer efficient for longer efficient,
the city should dissipate. Indeed, it seemed to be doing just that. 

However, by the 1980s, the growth of the corporate service sector and
the return to the city of high status groups confounded expectations
and demanded new explanations of urban form. Two main theories have
developed. First, that some industries require in-person coordination
between firms in order to thrive, so firms location near one
another. Second, some people live in cities in order to enact
lifestyles that are particularly available in cities. 

As we will see, these are both theories of mutualism. Mutualism is
where the presence of group A benefits group B and the presence of
group B benefits group A. 

\subsection*{Clustering}
Within urban sociology, the most prominent proponents of firm-level
mutualism have been Saskia Sassen and Michael Porter.

Saskia Sassen's work on global cities makes two basic moves. First,
she extends the Donald Bogue's concept of metropolitan dominance to a
globalized economy. As the size and scale of international production
expands, she argues, there is a complementary increase in centralized
administrative functions. Second, and going beyond Bogue, she argues
that those administrative functions should be concentrated in certain
sites because of these administrative functions are the joint product
of multiple firms, which requires intensive coordination facilitated
by proximity.

``At the current stage of technical development, having immediate and
simulataneous access to the pertinent experts is still the most
effective way to operate, especially when dealing with a highly
complex product.''\footnote{95}

In other words, for the firms in this field of production, the
physical presence of complementary firms provides a business
advantage.

Michael Porte identifies the same dynamic in a number of other
industries...

\subsection{Lifestyle}
Like the earlier theorists, for Sassen and Porter, firms are the prime
movers for the growth and form of cities. The other line of theorizing
the post-industrial city has emphasized residents and their
preferences. Initially, work in this line saw a new types of status
groups choosing city life as a form of consumption. Terry Clark and
foo's recent work goes beyond this to contemplate the co-production of
types of people and types of places. This is a mutualism between
people, between firms, and between people and firms.

Richard Florida's work has been typical of the main lifestyle
arguments. In his model, the creative class choose a place to live
based upon the amenability of place. High value firms and innovation
and economic growth follow the people. 

Most lifestyle theories, Florida included, may explain the relative
growth or decline of certain cities and regions, but not urban
agglomeration itself. In addition to arguing that, a. the a type of
worker chooses to live in a place based on lifestyle considerations,
and b. there is a stable elective affinity between that type of worker
and preference for a particular way of life; a theory of urban
agglomeration would also need to argue c. that places that attract
members of the creative class become even more attractive to members
of the class. 

For lifestyle to explain agglomeration, there must be an explanation of
why urban agglomerations are especially amenable to the crucial
lifestyle. Claude Fischer's subcultural theory provides such an
account. In this theory, cities stimulate expressive
unconventionality because of they stimulate subcultures. Fisher
enumerates many mechanisms for why cities generate and strengthen
subculture and why a diversity of subcultures, but they are first or
second order effects size or growth. In his account, unconventionality
is an effect of agglomeration not a cause.

In their recent work on ``scenes'', Terry Clark and Daniel Silver
married a theories of subcultures and lifestyle based residential
choice to form the first, explicit theory of urban agglomeration
driven by lifestyle. In their account, all people in advanced
economies are increasingly choosing a place to live in order to choose
a scene where they can express their aesthetics. A place's scene is
made by the current residents, both directly and indirectly through
the retail institutions that serve residents' lifestyles. When more of the
right type of people are in a scene, there are more opportunities for
expressive coaction and more institutions serving as venues for
expressive action.

\subsection*{Mutualism and Space}
Mutualistic interactions 





\section*{Whats left to say}



\stiglitz idea of construction


