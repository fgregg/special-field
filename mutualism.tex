\section*{Mutualism}
If the survival of firms depends on the place based costs of
production and transportation, then changes in the geography of those
costs will change the distribution of those types of firms. In the
early industrial period, transportation costs dominated other
locational factors, leading to the concentration of industry in key
nodes of the transportation. After the second world war, rising wages,
trucking, containerization, and new manufacturing techniques
undermined earlier advantages of the industrial city. Total
transportation costs became a secondary or tertiary consideration, and
the manufacturing firms first left the city often, later, the country.

For the theorists of the industrial city, the locational efficiencies
for firms were the fundamental reason for the city and its
growth. When the city was no longer efficient for longer efficient,
the city should dissipate. Indeed, it seemed to be doing just that. 

However, by the 1980s, the growth of the corporate service sector and
the return to the city of high status groups confounded expectations
and demanded new explanations of urban form. Two main theories have
developed. First, that some industries require in-person coordination
between firms in order to thrive, so firms location near one
another. Second, some people live in cities in order to enact
lifestyles that are particularly available in cities. 

As we will see, these are both theories of mutualism. Mutualism is
where the presence of group A benefits group B and the presence of
group B benefits group A. 

\subsection*{Clustering and Craft}
Within urban sociology, the most prominent proponents of firm-level
mutualism have been Saskia Sassen and Michael Porter.

Saskia Sassen's work on global cities makes two basic moves. First,
she extends the Donald Bogue's concept of metropolitan dominance to a
globalized economy.\cite{bogue_structure_1988} As the size and scale of international
production expands, she argues, there is a complementary increase in
centralized administrative functions. Second, and going beyond Bogue,
she argues that those administrative functions should be concentrated
in certain sites because of these administrative functions are the
joint product of multiple firms, which requires intensive coordination
facilitated by proximity: ``At the current stage of technical
development, having immediate and simultaneous access to the pertinent
experts is still the most effective way to operate, especially when
dealing with a highly complex product.''\cite{sassen_cities_2011}

Sassens argument about how proximity facilitates the production of
complicated and innovative services is a specific case of generic
argument that Michael Porter makes about ``industry clusters.''
Geographic clustering of an industry can, he argues, provide a number
of benefits for international competition. A concentration of
competitors, suppliers and customers can lead to specialization and
efficiency. Suppliers and competitors can a better 
communicate and coordinate the production of new products and
services. Firms can better surveil each
other and understand the overall state of the industry, allowing for
firms to more quickly react to changes and identify
opportunities.\cite{porter_competitive_1998}

The economic engines of post-industrial cities similar to
pre-industrial craft work. The work is variable, creative, unique, and
often customized for the client. As Stinchcombe predicted for this
type of work, the production is largely administed along craft
lines.\cite{stinchcombe_bureaucratic_1959} That is, the work is accomplished by a
network of specialist firms, that coordinate extensively but preserve
their own jurisdictions of work. The skill and knowledge necessary for
production is not codified into rules or operating procedures, but
held by expert workers that spend many years developing their skills
and knowledge, often in an apprentice system. 


\subsection*{Lifestyle}
Like earlier theorists, Sassen and Porter see firms as the prime
movers for the cities. However, other, new theories of cities have
elevated the resident as the driver of urban concentration and
growth. Early work in this line saw a new types of status groups
choosing city life as a form of consumption. Terry Clark and Dan
Silver's recent work goes beyond this to contemplate the co-production
of types of people and types of places. This is a mutualism between
people, between firms, and between people and firms.

Richard Florida's work has been typical of the most lifestyle
arguments. In his model, the workers with the human capital necessary
for post-industrial production tend to like living places that are
diverse, tolerant, and creative. These workers choose a place to live
based upon those preferences, not primarily upon where they can get a
job. High value firms and innovation and economic growth follow the
people.\cite{florida_rise_2014,clark_city_2011,berry_human_1974}

This is not just a reversal of the firm-centric models of the
industrial city, it's is also incompatible of with the earlier
status-based theories of residential selection. Those status based
theories assumed that there was a relatively stable status order based
on race, ethnicity, and occupation (and maybe caste, tribe, or
religion in other countries). People might not express their
residential preferences in terms of status, but their choices
effectively revealed a status preference. The lifestyle arguments
contradict this, arguing that an important group of people want to
live where it's fun, exciting, diverse. If the city or neighborhood is
gritty--poor and rundown, then that can be good too.

The lifestyle argument does not reject the theory that people choose
where to live in order to minimize the social distance between
themselves and the population they wish to emulate, just that that
social status is not the primary dimension for residential
selection.\footnote{Life cycle may help explain some of these
  effects. Young people can enjoy the gritty city, but move some place
  more socially appropriate when they want to have children and pass
  on their status.\cite{berry_human_1974}} 

While the lifestyle arguments flip the primary actor from the classic
firm-centric models of economy geography, the ultimate logic is the
same. A place has some endowment of resources, leading to differential
growth, and thus relative concentration of people and firms in 
privileged places.

Similar to that classic economic geography, the connection between a
place's growth and its endowments are not always clear. Is a place
large because it's diverse, creative, or tolerant. Or is it diverse,
creative, or tolerant because it is large.\footnote{Cronon does an
  excellent job describing how a place's natural advantages for
  transportation are often the work of the work of men.\cite{cronon_natures_1992}}

This connection matters a great deal, because it will determine both
the extent of urban concentration and the extent of segregation of
different ways of life in different cities. If the lifestyle
endowments are relatively fixed, then lifestyle based growth will
likely be self-limiting. The higher costs of land and other
disadvantages of concentration will eventually cancel out the benefits
of the endowment.

If the lifestyle endowments increase with growth, then the ceiling of
urban concentration is much higher. Claude Fischer's subcultural
theory describes one such positive feedback loop..

In this theory, cities stimulate expressive unconventionality because
of they stimulate subcultures. Fisher enumerates many mechanisms for
why cities generate and strengthen subculture and why a diversity of
subcultures, but they are first or second order effects size or
growth.\cite{fischer_subcultural_1995}

In Ficher's account, unconventionality is an effect of agglomeration
not a cause. But if we couple it with lifestyle arguments, we have a
positive feedback cycle. High human capital workers are attracted to
cities where they can express an unconventional lifestyle. This
growth, plus the second order growth due to the economic activities of
the firms that follow, make the city larger, which leads to greater
unconventionality. 

This feedback cycle is particular to people who value
unconventionality, and the firms that depend on them. It is also an
indirect cycle. City size mediates the link between lifestyle
attractions of unconventionality and the current population. City
population, not the composition of that population, produces
unconventionality. 

In their recent work on scenes, Terry Clark and Daniel Silver
argue that residential patterns are increasingly driven by much more
direct feedback cycles between the current population of people and
firms and lifestyle attractions. They also attend to a much wider
space of lifestyle that can drive residential choice.\cite{silver_scenescapes_2016}

All people in advanced economies, not just yuppies and neobohemians,
are increasingly choosing a place to live based upon lifestyle
considerations. In their concept of scenes, a place to live is a place
to enact aesthetics values. A place has a certain scene, if that place
has venues and co-actors that support the enactment of that scene's
aesthetics. People with aesthetic affinity are drawn to scenes where
they can work out their styling of life. An increased density of
co-actors strengthens a scene directly and indirectly, through the
growth of scene venues.

While Clark and Silver focus on retail establishments as the key
venues for scenes, post-industrial firms are also a venue for
aesthetics. The craft-like nature of post-industrial production
depends upon workers who have made long-term commitments to their work
to develop the requisite skill and expertise. That long encounters
shapes workers' aesthetics. Conversely, firms competing for critical
workers often compete on accomodation of lifestyle. Investment banks
and nightclubs are both part of one scene. Computer software and
pinball clubs are part of another.

In a fuller version of Clark and Silver's argument, post-industrial
workers, post-industrial firms, and retail venues are all part single
scene that jointly produce certain types of people, products, and
places. In turn, these scenes selectively attract people and firms
that can enhance that coproduction. 

\subsection*{Spatial Dynamics of Mutualistic Interactions}
Mutualistic interactions, all else being equal, lead to unlimited
growth and concentration. Every compatible new firm or immigrant,
improves the conditions for everyone else. A person or firm that
always prefer to occupy the city with the larger concentration of
mutualistic co-actors.

Mutualism also creates diversity between places and homogeneity within
places. Mutualism is ultimately a network effect. The benefit of place
is a function of the number of compatible people and organization in
that place. This suggests that cities should not attempt to compete
with other cities for mutualistic industries or scenes unless they can
prevent exit or have comparable current endowments with the current
lead city.  Cities should instead find a mutualistic industry that is
currently not well serviced by another city and foster that.

As that happens, by design or accident, every city should become the
best place for certain kind of firms or people. Cities should diverge
in their industry and populations, while becoming more homogenous
internally. 
