\section*{Competition} 
In the late 19th century and early 20th century, American urban
sociologists grappled with the unprecedented size and rapid growth of
the industrial city. Their main explanation was that economic
geography of transportation costs selected firms located at certain site;
sites that would swell into great cities.

Technological developments in shipping, canal building, and the rail
line had made long-distance transport much cheaper, but short-distance
transport remained expensive. Firms located near the central nodes of
the long-distance transport networks had much lower total costs of
transport than their backwater competitors. Economic competition
between firms in the newly national market would, therefore, tend to
select firms located at the best sites. As firms concentrated into a
few sites, their employees---the city's population---would follow.

In equilibrium, economic competition would minimize the total costs
of transportation for the entire economy.  Within a city, economic
competition would also result in the most accessible, central places
being occupied by the organizations that most benefited from
access the city.

\subsection*{Distributions of Settlements and Economic Geography}
Sociologists have mainly left the task of explaining the size and
locations of settlements to geographers and geographers have
historically focused on two factors: labor costs and transportation
costs of raw material and finished goods.\cite{firey_land_1947,mckenzie_ecological_1924}

The geographic distribution of raw materials, transportation network,
population, and labor prices combine to form a geography of costs of
operating a firm. Within this economic geography, certain locations
confer advantage to their resident firms who win out against their
poorly positioned competitors. Advantaged locations would swell while
the backwaters dwindled.\cite{harris_nature_1945}

Not all firms would prefer the exact same type of location. Firms that
transformed bulky raw material into much lighter finished goods should
locate near the source of raw materials. Firms that distributed
perishable goods should locate near their consumers. Whatever the type
of firm though, the distribution of people and raw materials and the
transit network produced locations that would place that type of firm
at a competitive advantage.

The early industrial city grew so quickly because, as locations, they
provided enormous advantages to industrial firms competing in a newly,
national market. The network of first canals and then railroads
slashed the costs of transportation at central nodes of the
network. Manufacturing firms located at those nodes could outcompete
firms located in bypassed towns and even local craft works. Because
short-distance haulage remained expensive, firms clustered around the
transit nodes, and workers clustered around the
firms.\cite{hawley_urban_1981,cronon_natures_1992}

\subsection*{Distributions within Settlements}
Within cities, short-distance transportation costs also create an economic
geography. Across a city, the distribution of people and organizations
interacts with the transportation networks to produce areas of greater
accessibility. The most accessible place should be dominated by the
type of users who benefit most from accessibility relative to their
unit costs of land.\cite{alonso_theory_1960} The unit costs, in turn, were the
outcome of market competition between firms and residents for the
accessibility.

In a rare point of consensus, the early 20th century urban
sociologists agreed that commercial firms should and could outbid
residents or other firms for the most accessible parts of the
city.\cite{park_growth_1984,harris_nature_1945} Even Firey allowed that
it was economically rational for large department stores to dominate
downtowns.\cite{firey_land_1947}

Beyond the central business district, there were many locally maximal
accessible locations, typically at the intersection of major streets
or transportation lines. Smaller, local commercial firms should occupy
these spaces, forming miniature central business district.\cite{hoyt_one_1970}

Manufacturers could also form distinct nuclei of populations. Oriented
to national markets, they would choose locations that minimized their
long distance transport costs and had ample land for large
plants. These locations could be far from the center of population,
but if employment was concentrated, employees would follow to reside
in its proximity.

While sociologists more or less agreed that competition for
accessibility would determine the locations of commerce and industry,
there was substantial disagreement about how people should distribute
themselves in the places left to them.

\subsection*{Population Ecology}
The competitive theories of industrial city are special case of the
population ecology model, most associate with Hannan and Freeman.  In
economic geography, the \emph{spatial distribution} of firms is the
result of the competition and selection of firms who have a
\emph{locational} advantage over competitors facing the same
market. In the classic formulation of population ecology, the
\emph{distribution of populatoins} of different types of firms is the
result of competition and selection of firm who have a
\emph{strategic} advantage over competitors facing the same market. In
both cases, the distribution of firms is the result of selection
among competitors by an environment.\cite{hannan_population_1977}

\subsection*{Spatial Dynamics of Competitive Interactions}
As a field, we have mainly attended to competitive interactions as a
component mechanism for how an environment selects
populations. Competitive interactions, though, do have their own, internal,
spatial dynamic. All else being equal, competitive interactions should
lead to the spatial dispersion of populations, as actors will
benefit from isolating themselves from competitors.

In a competitive interaction, group A would be better off if group B
was removed from its presence, and vice versa. Competing groups would
be best off if they were physically separated.  In real estate
markets, firms competing for accessibility bids up the land rent of
the central place leading most organizations and people to live
elsewhere, where the accessibility is lower but so is price.

If a particular site provide advantages to populations that locate
there, competition both explains why the population concentrates at
that site---they are better able to reproduce or survive than
population located elsewhere. Competition also explains why the entire
population doesn't concentrate as close as they can to a single
point. At some level of concentration, the disadvantages of competition
outweigh the initial advantages of the site.

In economic terms, this is the same idea, from Alonso, that
competitive farmers should all make the same ``normal profit''.
Regardlesss of the fertility or location of the land, since more
fertile land, closer to market will be priced higher than marginal
distant land.\cite{alonso_theory_1960}

Given an economic geography that provides a fixed economic advantage
for populations of firms at a city site, competitive interactions explain
city size. The advantages of the site produces concentration but the
effects of competition limit concentration. If those initial
advantages of place disappear, the centrifugal dynamic of competition
should quickly reduce concentration.

The initial advantages of the industrial city eroded throughout the
20th Century. First, the automobile and then the highway system
reduced the costs of short-distance transport, allowing manufacturing
firms to move away from the depots of long-distance transport. While
the earlier communication and transportation innovations of the
industrial age produced a national or continental market for
manufacturing goods, further developments in communication and
transportation, particularly containerization, reduced the costs of
international transport and put manufacturing firms in global
competition. American industrial cities no longer provided a relative
advantage to manufacturing firms, and the populations of firms
dwindled. The population of people did as well.






